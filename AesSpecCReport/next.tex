\section{Next Steps} \label{sec:next}

%Much work remains to expand \FrameworkName into a robust CPS assertion framework.  
In the short term, we plan to support more sensor types and services to enhance \FrameworkName's ability to collect physical data.  Table~\ref{tab:sensorList} shows the list of sensors that are already supported by \FrameworkName (denoted by $\diamondsuit$) and the sensors planned for future support.
While our current implementation only supports in-lined assertions, we plan to implement support for continuous and temporally-tolerant assertions. % in the near future.  %For example, a user may define a threshold discrepancy between the logical and physical states to tolerate minor changes in measurement such as the margin of 5{\tt mm} error in the position of a robot. 
%Furthermore, we intend to support assertions with timing constraints; such constraints would force \FrameworkName to either evaluate the assertion within a certain time frame or cancel the assertion if the deadline cannot be met.  For example, the assertion in Figure~\ref{fig:assertion-example-framework}, lines 6-7, can be modified to be: \begin{quote}\texttt{\textbf{CPSAssertAsync}}(\ldots) \texttt{\textbf{within}} 5 min;\end{quote} \noindent to indicate that the asynchronous assertion must be evaluated within 5 minutes, or not evaluated at all. 
% Timing constraints can reduce the number of pending assertions to be evaluated, as well as modeling how real-world values can become irrelevant after a certain time period.
% Liang removed this because this is handled entirely within the services that convert physical to logical state.
%Another enhancement is the ability to fuse sensor data from multiple sources.  For example, instead of using a single sensor to measure the room temperature, \FrameworkName could use the average over multiple sensors in the room. These data aggregation techniques can be integrated into \FrameworkName by hiding them beneath the API.

In the long run, we plan on providing an off-line debugging tool that enables developers to investigate runtime traces captured by \FrameworkName to identify the root cause(s) of failed assertions. 
%In addition, recorded traces can be used to measure test coverage with respect to {\em physical environments} as opposed to {\em code coverage}: e.g., %the coverage analysis may conclude 
%``the robot only explored the kitchen, so we are unsure if the program will run correctly in the living room.'' 
We also envision using CPS assertions as a basis for automatically correcting CPS system behavior in the future. By differencing expected physical states and actual physical states, a novel patch generation algorithm can modify the program to make physical states match the corresponding logical states. 

% MK moved the following because did not understand. %Such tool must account for the fact that \FrameworkName assertions may be evaluated long after they were declared, resulting in extended fault propagation.  %\FrameworkName provides assertions that detect faults, but actually localizing the source of the faults will require backtracking.  Traditional techniques must be enhanced to account for the fact that the assertions in \FrameworkName may be asynchronous, meaning faults will have more time to propagate. 

% MK dropped the following. our text is very repetitive. Enabling developers to understand, debug, and correct cyber-physical system behavior easily is essential. Our early experience of building CPS systems show that a logically correct program may not result in a physically correct system due to unexpected properties of the physical world. For this reason, an assertion framework like \FrameworkName is necessary and valuable for developers to check the alignment between a program's logical state and the environment's physical state.
